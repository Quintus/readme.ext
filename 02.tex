\chapter{Grundwissen}
\label{cha:grundwissen}

In C haben Variablen Typen und Daten nicht. Im Gegensatz dazu haben
Ruby-Variablen keinen festen Typ, aber dafür die Daten selbst, sodass
jegliche Daten zwischen den beiden Sprachen konvertiert werden
müssen.

Ruby-Daten werden in C durch den Typ \VALUE beschrieben. Jedes
\VALUE-Objekt besitzt einen Datentyp \trans{der der Ruby-Klasse
  entspricht}.

Um die C-Daten eines \VALUEs zu erhalten, müssen Sie: 

\begin{enumerate}
\item Den Datentyp des \VALUEs herausfinden
\item Den \VALUE in C-Daten konvertieren
\end{enumerate}

Eine Konversion zu einem falschen Datentyp kann schwerwiegende Folgen
haben.

\section{Datentypen}
\label{sec:datentypen}

Der Ruby-Interpreter kennt die in Tabelle \ref{tab:datentypen}
definierten Datentypen. Die meisten dieser Typen werden durch
C-Strukturen (\verb+struct+s) beschrieben.

\begin{table}[hbtp]\centering
  \begin{tabular}{>{\ttfamily}ll>{\ttfamily}l}
    \textbf{\rmfamily Typ} & \textbf{Beschreibung}   & \textbf{\rmfamily Ruby-Klasse} \\
    \hline
    \multicolumn{3}{c}{Wahrheitswerte}\\
    \hline
    T\_FALSE               & \verb+false+            & FalseClass \\
    T\_NIL                 & \verb+nil+              & NilClass \\
    T\_TRUE                & \verb+true+             & TrueClass \\
    \hline
    \multicolumn{3}{c}{Objekte}\\
    \hline
    T\_ARRAY               & Array                   & Array \\
    T\_BIGNUM              & Mutli-Precision-Integer & Bignum \\
    T\_CLASS               & Klasse                  & Class \\
    T\_COMPLEX             & Komplexe Zahl           & Complex \\
    T\_FILE                & IO-Stream               & IO \\
    T\_FIXNUM              & 31- oder 63-Bit-Integer & Fixnum \\
    T\_FLOAT               & Gleitkommazahl          & Float \\
    T\_HASH                & Assozatives Array       & Hash \\
    T\_MODULE              & Modul                   & Module \\
    T\_OBJECT              & Gewöhnliches Objekt     & \\
    T\_RATIONAL            & Rationale Zahl          & Rational \\
    T\_REGEXP              & Regulärer Ausdruck      & Regexp\\
    T\_STRING              & String                  & String \\
    T\_STRUCT              & (Ruby-)Struktur         & Struct \\
    T\_SYMBOL              & Symbol                  & Symbol \\
    \hline
    \multicolumn{3}{c}{Anderes}\\
    \hline
    T\_DATA                & Daten                   & \\
    \hline
    \multicolumn{3}{c}{Intern genutzte Typen}\\
    \hline
    T\_ICLASS &&\\
    T\_MATCH  &&\\
    T\_NODE   &&\\
    T\_UNDEF  &&\\
    T\_ZOMBIE
  \end{tabular}

  \vspace{\baselineskip}

  Alle Typen sind gewöhnliche konstante Zahlen.
  \caption{Von Ruby verwendete Datentypen}
  \label{tab:datentypen}
\end{table}

\section{Den Datentyp eines \VALUEs überprüfen}
\label{sec:pruefen}

Das Makro \verb+TYPE()+, das in der Datei \path{ruby.h} definiert ist,
ermöglicht die Überprüfung des Datentyps eines \VALUEs. \verb+TYPE()+
gibt eines der in Tabelle \ref{tab:datentypen} angegebenen
\verb+T_XXXX+-Makros (es handelt sich hierbei ausschließlich um
Zahlen) zurück. Um verschiedene Datentypen zu behandeln, wird Ihr Code
vermutlich so oder ähnlich aussehen:

\begin{lstlisting}
switch (TYPE(obj)) {
  case T_FIXNUM:
    /* Fixnum behandeln */
    break;
  case T_STRING:
    /* String behandeln*/
    break;
  case T_ARRAY:
    /* Array behandeln*/
    break;
  default:
    /* Exception werfen */
    rb_raise(rb_eTypeError, "not valid value");
    break;
}
\end{lstlisting}

Daneben gibt es noch eine Funktion, um den Datentyp zu überprüfen:

\begin{lstlisting}
  void Check_Type(VALUE value, int type)
\end{lstlisting}

\noindent Sie wirft eine Exception \trans{der Klasse
  \fverb{TypeError}}, wenn der \VALUE nicht den angegeben Typ besitzt.

Für Fixnums und \verb+nil+ gibt es schnellere Makros:

\begin{lstlisting}
FIXNUM_P(obj)
NIL_P(obj)
\end{lstlisting}

\trans{Darüber hinaus gibt es ein weiteres Makro namens
  \fverb{RTEST()}, welches das Verhalten eines üblichen
  Ruby-\fverb{if}-Statements ohne weitere Bedingungen (also einen
  Existenzcheck) nachahmt. Dieses Makro gibt für \fverb{nil} und
  \fverb{false} einen falschen und für alle anderen Objekte einen
  wahren Wert zurück.}

\section{Einen \VALUE in C-Daten konvertieren}
\label{sec:value2c}

Die Daten für die Typen \verb+T_NIL+, \verb+T_FALSE+ und \verb+T_TRUE+
sind jeweils \verb+nil+, \verb+false+ sowie \verb+true+. In C werden
sie durch die folgenden Konstanten beschrieben: \verb+Qnil+,
\verb+Qfalse+ und \verb+Qtrue+. Beachten Sie, dass \verb+Qfalse+ zwar
auch im C-Sinne falsch ist (d.i. \verb+0+), nicht jedoch \verb+Qnil+.

\begin{table}[btp]
  \centering\ttfamily
  \begin{tabular}{lll}
    \textbf{\rmfamily Datentyp} & \textbf{\rmfamily Inhalt in Ruby} & \textbf{\rmfamily C-Konstante}\\
    \hline
    T\_FALSE & false & Qfalse \\
    T\_NIL   & nil   & Qnil \\
    T\_TRUE  & true  & Qtrue \\
  \end{tabular}
  \caption{Übersicht über die Wahrheitswerte}
  \label{tab:wahrheitswerte}
\end{table}

Die Daten des \verb+T_FIXNUM+-Datentyps sind \trans{je nach
  Prozessorarchitektur} feste 31- oder 63-Bit-Integers. Diese Größe
hängt von der Größe des C-Typs \verb+long+ ab; ist \verb+long+ 32 Bit
groß, so enthält \verb+T_FIXNUM+ 31 Bit, ist es indessen 64 Bit groß,
so enthält \verb+T_FIXNUM+ 63 Bit. \trans{Auf Seite
  \pageref{sec:c2value} wird erläutert, wofür das hier fehlende Bit
  verwendet wird.}

\verb+T_FIXNUM+ kann mithilfe der Makros \verb+FIX2INT()+ und
\verb+FIX2LONG()+ in einen C-Integer konvertiert werden. Obwohl Sie
vor deren Gebrauch prüfen müssen, ob die Daten tatsächlich ein
\verb+Fixnum+ sind, sind sie schneller. \verb+FIX2LONG()+ wirft
niemals Exceptions, aber \verb+FIX2INT()+ wirft einen
\verb+RangeError+ wenn das Ergebnis kleiner oder größer als der
Datenbereich von \verb+int+ ist.

Außerdem gibt es noch \verb+NUM2INT()+ sowie \verb+NUM2LONG()+, die
jede Ruby-Zahl in einen C-Integer konvertieren. Diese Makros enthalten
zudem eine Typüberprüfung, weshalb sie eine Exception werfen, wenn die
Konvertierung fehlschlägt. Um \verb+double float+s zu erhalten, kann
man \verb+NUM2DBL()+ auf dieselbe Art und Weise verwenden.

Sie können die Makros \verb+StringValue()+ und \verb+StringValuePtr()+
verwenden, um einen \verb+char*+ aus einem \VALUE zu erhalten.
\verb+StringValue(var)+ ersetzt den Wert von \verb+var+ mit dem
Ergebnis von \lstinline+var.to_str()+. \verb+StringValuePtr(var)+
macht das Gleiche, gibt jedoch die \verb+char*+\-Re\-prä\-sen\-ta\-tion
von \verb+var+ zurück. Beide Makros werden den Ersetzungsschritt
auslassen, falls \verb+var+ bereits ein Ruby-String ist.  Ebenfalls
wichtig ist, dass diese Makros nur Lvalues als Argumente annehmen,
sodass sie den Wert von \verb+var+ in-place ändern können.

Sie können auch ein Makro namens \verb+StringValueCStr()+
benutzen. Prinzipiell ist das das Gleiche wie \verb+StringValuePtr()+,
fügt aber zusätzlich noch ein extra \verb+NUL+-Zeichen an das Ende
an. Wenn das Ergebnis \trans{der Konvertierung mithilfe von
  \fverb{to\_str}} bereits ein \verb+NUL+-Zeichen enthält, wirft
dieses Makro einen \verb+ArgumentError+. \verb+StringValuePtr()+
garantiert das eine, abschließende \verb+NUL+ nicht, und es kann
passieren, dass das Ergebnis mehrere \verb+NUL+-Zeichen enthält.

\begin{table}[btp]
  \centering
  \begin{tabularx}{\linewidth}{>{\ttfamily}l>{\ttfamily}lX}
    \textbf{\rmfamily Makro} & \textbf{\rmfamily Rückgabewert} & \textbf{Beschreibung}\\
    \hline

    StringValue(val) & VALUE & Äquivalent zu \lstinline+val.to_str+ in
    Ruby.\\

    StringValuePtr(val) & char* & Wie \verb+StringValue(val)+, aber
    gibt einen C-String zurück. Das Ergebnis kann kein, ein oder
    mehrere \verb+NUL+-Zeichen enthalten.\\

    StringValueCStr(val) & char* & Wie \verb+StringValuePtr+, aber
    garantiert genau \emph{ein} \verb+NUL+-Zeichen im gesamten String,
    und zwar am Ende.
  \end{tabularx}
  \caption{Möglichkeiten der Konvertierung eines Ruby-Strings in einen C-String}
  \label{tab:stringkonv}
\end{table}

Andere Datentypen haben passende C-Strukturen, beispielsweise der
\verb+RArray+-Struct für \verb+T_ARRAY+, etc. Ein \VALUE, der eine
passende C-Struktur besitzt, kann zu einem Pointer auf den Struct
gecastet werden. Makros, die einen solchen Cast durchführen, haben die
Form \verb+RXXXX+, beispielsweise \verb+RARRAY(obj)+. Siehe dazu die
Datei \path{ruby.h}.

Es gibt ein paar Zugriffsmakros für Struct-Members, zum Beispiel
\verb+RSTRING_LEN(s)+, um die Länge eines Ruby-String-Objekts zu
erhalten. Auf den allokierten Speicher kann mithilfe von
\verb+RSTRING_PTR(str)+ zugegriffen werden. Für Arrays existieren
dementsprechend \verb+RARRAY_LEN(ary)+ und \verb+RARRAY_PTR(ary)+.

\begin{notice}
  Ändern Sie niemals direkt einen Wert einer Struktur, wenn Sie nicht
  für das Ergebnis geradestehen wollen. Solche Sachen sind oftmals
  Grund für einige interessante Bugs.
\end{notice}

\section{Konvertieren von C-Daten in \VALUEs}
\label{sec:c2value}

Um C-Daten in Ruby-\VALUEs zu konvertieren gibt es zwei Möglichkeiten:

\begin{description}
\item[\fverb{Fixnum}] 1 Bit nach links verschieben, dann das LSB
  setzen.
\item[Andere \VALUE-Pointer] Nach \VALUE casten.
\end{description}

Ob ein \VALUE ein Pointer ist oder nicht, können Sie
durch Überprüfung des LSBs feststellen.

Bedenken Sie außerdem, dass Ruby nicht jedem beliebigen Pointer
erlaubt, ein \VALUE zu sein; es muss sich um Pointer auf Ruby bekannte
Strukturen handeln. Diese Strukturen sind in der Datei \path{ruby.h}
defininiert.

Um C-Zahlen in Ruby-Zahlen zu konvertieren, können Sie diese Makros
benutzen:

\begin{description}
\item[\fverb{INT2FIX()}] für Integers mit einer Länge von 31 Bit oder
  weniger.
\item[\fverb{INT2NUM()}] für beliebig große Integers.
\end{description}

\verb+INT2NUM()+ konvertiert einen Integer in ein Bignum, wenn er
außerhalb des Fixnum-Bereichs liegt, ist aber ein bisschen langsamer.

\section{Verändern von Ruby-Daten}
\label{sec:ruby-daten-aendern}

Wie ich bereits beschrieben habe, empfehle ich nicht, die interne
Struktur eines Objekts zu verändern. Um Objekte zu bearbeiten, nutzen
Sie bitte die Funktionen, die Ihnen der Ruby-Interpreter bereitstellt.
Viele (aber nicht alle) dieser nützlichen Funktionen sind in Tabelle
\ref{tab:rubyfuncs} aufgeführt.

\begin{notice}
  Übergibt man diesen Funktionen ein Objekt eines anderen Typs als
  jenem, den sie erwarten, können sie den Ruby-Interpreter zum Absturz
  bringen \trans{und das ist nicht nur bei Arrays so\footnote{Im
      Originaldokument wurden an dieser Stelle lediglich Arrays
      angesprochen.}.}.
\end{notice}

\begin{longtable}{p{0.5\textwidth}p{0.5\textwidth}}
  \caption{Einige nützliche Funktionen für Ruby-\VALUEs}\label{tab:rubyfuncs}\\
  \textbf{Funktion und Aufruf} & \textbf{Beschreibung}\\ \hline
\endfirsthead
  \textbf{Funktion und Aufruf} & \textbf{Beschreibung}\\ \hline
\endhead
  \hline \multicolumn{2}{r}{\emph{Fortsetzung auf nächster Seite}}
\endfoot
  \hline
\endlastfoot
  \multicolumn{2}{c}{String-Funktionen}\\
  \hline
  \begin{lstlisting}
rb_str_new(const char *ptr,
  long len)
  \end{lstlisting}&
  \begin{flushleft}
    Erstellt einen Ruby-String mit bestimmter Länge.
  \end{flushleft}\\

  \begin{lstlisting}
rb_str_new2(const char *ptr)
rb_str_new_cstr(const char *ptr)
  \end{lstlisting}&
  \begin{flushleft}
    Erstellt einen Ruby-String aus einen C-String. Es ist das
    Gleiche wie \lstinline+rb_str_new(ptr, strlen(ptr))+.
  \end{flushleft}\\

  \begin{lstlisting}
rb_enc_str_new(const char *ptr,
  long len, rb_encoding *enc)
  \end{lstlisting}&
  \begin{flushleft}
    Erstellt einen neuen String mit dem angegebenen Encoding
    \trans{\fverb{*ptr} sollte den String auch entsprechend kodiert
      enthalten, Ruby setzt nur das Encoding-Tag}.
  \end{flushleft}\\

  \begin{lstlisting}
rb_usascii_str_new(
  const char *ptr, long len) 
rb_usascii_str_new_cstr(
  const char *ptr)
  \end{lstlisting}&
  \begin{flushleft}
    Erstellt einen neuen String mit US-ASCII-Encoding.
  \end{flushleft}\\

  \begin{lstlisting}
rb_utf8_encoding()
  \end{lstlisting}&
  \begin{flushleft}
    Gibt das \verb+rb_encoding*+ für UTF-8 zurück. \transmark
  \end{flushleft}\\

  \begin{lstlisting}
rb_enc_find(const char* name)    
  \end{lstlisting}&
  \begin{flushleft}
    Gibt das \verb+rb_encoding*+ für \verb+name+
    zurück. \transmark
  \end{flushleft}\\

  \begin{lstlisting}
rb_str_export_to_enc(
  VALUE str, rb_encoding* enc)
  \end{lstlisting}&
  \begin{flushleft}
    Äquivalent zu \lstinline+str.encode(enc)+ in Ruby. \transmark
  \end{flushleft}\\

  \begin{lstlisting}
rb_tainted_str_new(
  const char *ptr, long len)
  \end{lstlisting}&
  \begin{flushleft}
    Erstellt einen Ruby-String, der \emph{tainted} ist. Strings, die
    aus externen Quellen kommen, sollten als \emph{tainted} markiert
    werden. \trans{Das gehört zu Rubys Sicherheitskonzept}
  \end{flushleft}\\

  \begin{lstlisting}
rb_tainted_str_new2(
  const char *ptr)
rb_tainted_str_new_cstr(
  const char *ptr)
  \end{lstlisting}&
  \begin{flushleft}
    Erstellt einen neuen \emph{tainted}-String. Funktioniert wie
    \verb+rb_str_new_cstr()+.
  \end{flushleft}\\

  \begin{lstlisting}
rb_sprintf(const char *format,
  ...)
rb_vsprintf(const char *format,
  va_list ap)
  \end{lstlisting}&
  \begin{flushleft}
    Erstellt einen neuen Ruby-String mithilfe des
    \man{printf}{3}-Formats.
  \end{flushleft}\\

  \begin{lstlisting}
rb_str_cat(VALUE str,
  const char *ptr, long len)
  \end{lstlisting}&
  \begin{flushleft}
    Hängt \verb+len+ Bytes von \verb+ptr+ an den Ruby-String an.
  \end{flushleft}\\

  \begin{lstlisting}
rb_str_cat2(VALUE str,
  const char* ptr)
  \end{lstlisting}&
  \begin{flushleft}
    Hängt einen C-String an einen Ruby-String an. Das ist das Gleiche
    wie \lstinline+rb_str_cat(str, ptr, strlen(ptr))+.
  \end{flushleft}\\

  \begin{lstlisting}
rb_str_catf(VALUE str,
  const char* format, ...)
rb_str_vcatf(VALUE str,
  const char* format, va_list ap)
  \end{lstlisting}&
  \begin{flushleft}
    Hängt den C-String und nachfolgende Argumente an den Ruby-String
    \verb+str+ an. Diese Funktionen sind jeweils äquivalent zu
    \lstinline+rb_str_cat2(str, rb_sprintf(format, ...))+ und
    \lstinline+rb_str_cat2(str, rb_vsprintf(format, ap))+.
  \end{flushleft}\\

  \begin{lstlisting}
rb_str_resize(VALUE str,
  long len)
  \end{lstlisting}&
  \begin{flushleft}
    Ändert die Länge des Ruby-Strings \verb+str+ auf \verb+len+
    Bytes und wirft eine Exception, wenn \verb+str+ nicht veränderbar
    ist. Die Länge von \verb+str+ muss im Vorhinein festgelegt
    werden. Ist \verb+len+ kleiner als die alte Länge, werden die
    Inhalte hinter \verb+len+ Bytes abgeschnitten, ist es größer als
    die alte Länge, so werden die Inhalte hinter \verb+len+ Bytes
    nicht erhalten und werden unbrauchbar sein \trans{Gemeint ist,
      dass bei der Vergrößerung des Strings keine Garantie für
      irgenwelche Inhalte hinter dem bisherigen Inhalt übernommen
      werden kann, abgesehen davon, dass da \emph{irgendwas} sein
      wird.}. Beachten Sie, dass der Rückgabewert von
    \verb+RSTRING_PTR(str)+ durch den Aufruf dieser Funktion ändern
    kann.
  \end{flushleft}\\

  \begin{lstlisting}
rb_str_set_len(VALUE str,
  long len)
  \end{lstlisting}&
  \begin{flushleft}
    Setzt die Länge des Ruby-Strings \verb+str+ und wirft eine
    Exception, wenn \verb+str+ nicht veränderbar ist. Diese Funktion
    garantiert den Bestand von bis zu \verb+len+ Bytes unter
    Missachtung von \verb+RSTRING_LEN(str)+. \verb+len+ darf die
    Kapazität von \verb+str+ nicht überschreiten.
  \end{flushleft}\\

  \hline
  \multicolumn{2}{c}{Array-Funktionen}\\
  \hline

  \begin{lstlisting}
rb_ary_new()
  \end{lstlisting}&
  \begin{flushleft}
    Erstellt ein Array ohne Inhalt.
  \end{flushleft}\\

  \begin{lstlisting}
rb_ary_new2(long len)
  \end{lstlisting}&
  \begin{flushleft}
    Erstellt ein Array ohne Inhalt, allokiert aber internen Speicher
    für \verb+len+ Elemente.
  \end{flushleft}\\

  \begin{lstlisting}
rb_ary_new3(long n, ...)
  \end{lstlisting}&
  \begin{flushleft}
    Erstellt ein \verb+n+-elementiges Array aus den Argumenten
    \trans{bei denen es sich um \VALUE-Objekte handeln muss.}.
  \end{flushleft}\\

  \begin{lstlisting}
rb_ary_new4(long n, VALUE *elts)
  \end{lstlisting}&
  \begin{flushleft}
    Erstellt aus einem C-Array ein \verb+n+-elementiges Ruby-Array.
  \end{flushleft}\\

  \begin{lstlisting}
rb_ary_to_ary(VALUE obj)
  \end{lstlisting}&
  \begin{flushleft}
    Konvertiert das Objekt in ein Array. Entspricht Object\#to\_ary.
  \end{flushleft}\\

  \begin{lstlisting}
rb_ary_aref(int argc,
  VALUE *argv, VALUE ary)
  \end{lstlisting}&
  \begin{flushleft}
    Äquivalent zu Array\#[].
  \end{flushleft}\\

  \begin{lstlisting}
rb_ary_entry(VALUE ary,
  long offset)
  \end{lstlisting}&
  \begin{flushleft}
    \lstinline+ary[offset]+.
  \end{flushleft}\\

  \begin{lstlisting}
rb_ary_subseq(VALUE ary,
  long beg, long len)
  \end{lstlisting}&
  \begin{flushleft}
    \lstinline+ary[beg, len]+.
  \end{flushleft}\\

  \begin{lstlisting}
rb_ary_push(VALUE ary,
  VALUE val)
  \end{lstlisting}&
  \begin{flushleft}
    Hängt \verb+val+ an \verb+ary+ an. \transmark
  \end{flushleft}\\

  \begin{lstlisting}
rb_ary_pop(VALUE ary)
  \end{lstlisting}&
  \begin{flushleft}
    Entfernt das letzte Element aus dem Array und gibt es
    zurück. \transmark
  \end{flushleft}\\

  \begin{lstlisting}
rb_ary_shift(VALUE ary)
  \end{lstlisting}&
  \begin{flushleft}
    Entfernt das erste Element aus dem Array und gibt es
    zurück. \transmark
  \end{flushleft}\\

  \begin{lstlisting}
rb_ary_unshift(VALUE ary,
  VALUE val)
  \end{lstlisting}&
  \begin{flushleft}
    Fügt \verb+val+ als das erste Element des Arrays ein; die
    restlichen Elemente werden um 1 nach hinten
    geschoben. \transmark
  \end{flushleft}\\
\end{longtable}


%%% Local Variables: 
%%% mode: latex
%%% TeX-master: "main"
%%% End: 
