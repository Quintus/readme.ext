\begin{titlepage}\sffamily
  \begingroup\Large README.EXT\endgroup\par
  C-Extensions für die Programmiersprache Ruby
\end{titlepage}

%\cleardoublepage

\begin{titlepage}\centering
  \begingroup\Huge{README.EXT}\endgroup\\
  \rule{\textwidth}{2pt}

  \vspace{\baselineskip}

  \noindent\begingroup\LARGE C-Extensions für die Programmiersprache Ruby\endgroup

  \vspace{0.5\baselineskip}

  \noindent\textit{Aus dem Englischen von Marvin Gülker (Quintus)}

  \vspace{\baselineskip}

  \begingroup\footnotesize
  2.\,Auflage Juni \the\year % FIXME: How to get the current localised
                             % month name?
  \endgroup

  \vspace{\baselineskip}

  \vfill

  \begin{minipage}{0.45\linewidth}\raggedright\footnotesize
    Original\\
    Copyright © 1995-2012 Yukihiro Matsumoto\\
    \href{http://www.ruby-lang.org/de/about/license.txt}{Rubys Lizenz}
  \end{minipage}
  \hfill
  \begin{minipage}{0.45\linewidth}\raggedleft\footnotesize
    Deutsche Übersetzung\\
    Copyright © 2010,2012 Marvin Gülker\\
    \href{http://creativecommons.org/licenses/by-sa/3.0/de/}{CC-BY-SA}
  \end{minipage}

  \newpage\thispagestyle{empty}

  \begin{flushleft}
    \vspace*{\fill}
    \noindent Serifenschrift: Libertine\\
    Serifenlose Schrift: \textsf{Biolinum}\\
    Typschrift: \texttt{Computer Modern Typewriter}

    \vspace{\baselineskip}

    \noindent Gesetzt mit dem Schriftsatzsystem \LaTeX.
  \end{flushleft}
\end{titlepage}

%%% Local Variables: 
%%% mode: latex
%%% TeX-master: "main"
%%% End: 
