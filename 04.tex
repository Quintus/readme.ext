\chapter{Informationen zwischen Ruby und C transferieren}
\label{cha:infos-ruby-c}

\section{Ruby-Konstanten, die von C aus abgerufen werden können}
\label{sec:ruby-konstanten-von-c}

Die folgenden Ruby-Konstanten können von C aus abgerufen werden:

\begin{itemize}
\item \verb+Qtrue+
\item \verb+Qfalse+
\end{itemize}

\noindent Dies sind boolsche Werte. \verb+Qfalse+ ist in C auch
einfach \verb+0+.

\begin{itemize}
\item \verb+Qnil+
\end{itemize}

\noindent Und das ist das Ruby-\verb+nil+ in C. Diese Informationen
können auch in Tabelle \ref{tab:wahrheitswerte} auf Seite
\pageref{tab:wahrheitswerte} gefunden werden.

\section{Globale Variablen zwischen Ruby und C}
\label{sec:glob-variablen-ruby-c}

Mithilfe von globalen Variablen können Werte sowohl von Ruby als auch
von C aus benutzt werden. Benutzen Sie die folgenden Funktionen, um
sie zu definieren:

\begin{lstlisting}
void rb_define_variable(const char *name, VALUE *var)
\end{lstlisting}

Diese Funktion definiert eine Variable, die von beiden Umgebungen
abrufbar ist.  Der Wert, auf den die Variable namens \verb+var+ zeigt,
kann dann von Ruby aus mithilfe einer globalen Variable namens
\verb+name+ \trans{mit \fverb{\$} davor} abgerufen werden.

Sie können read-only-Variablen (natürlich nur von Ruby aus nicht
schreibbar) mit dieser Funktion definieren:

\begin{lstlisting}
void rb_define_readonly_variable(const char *name, VALUE *var)
\end{lstlisting}

Es ist auch möglich, sogenannte \emph{Hook-Variablen} definieren. Wenn
eine solche Variable abgerufen oder ihr ein Wert zugewiesen wird, wird
entsprechend die \verb+getter+- oder \verb+setter+-Funktion
aufgerufen.

\begin{lstlisting}
void rb_define_hooked_variable(const char *name, VALUE *var,
                               VALUE (*getter)(),
                               void (*setter)())
\end{lstlisting}

\noindent Wenn Sie nur eine der beiden Funktionen nutzen wollen,
setzen Sie einfach 0 für den Hook, den Sie nicht benötigen. Sind beide
Hooks 0, dann macht diese Funktion das Gleiche wie
\verb+rb_define_variable()+.

Die Prototypen der \verb+getter+- und \verb+setter+-Funk\-tion\-en
sehen wie folgt aus:

\begin{lstlisting}
VALUE (*getter)(ID id, VALUE *var);
void (*setter)(VALUE val, ID id, VALUE *var);
\end{lstlisting}

Es besteht ebenfalls die Möglichkeit, eine in Ruby globale-Variable
ohne eine entsprechende C-Variable zu definieren. Der Wert einer
solchen Variablen wird nur durch Set- sowie Get-Hooks gesetzt und
ausgelesen.

\begin{lstlisting}
void rb_define_virtual_variable(const char *name,
                                VALUE (*getter)(),
                                void (*setter)())
\end{lstlisting}

Die Prototypen der \verb+getter+- und \verb+setter+-Funk\-tion\-en
sehen so aus:

\begin{lstlisting}
VALUE (*getter)(ID id);
void (*setter)(VALUE val, ID id);
\end{lstlisting}

\section{C-Daten in ein Ruby-Objekt einbinden}
\label{sec:c-in-ruby}

Um einen C-Pointer zu wrappen und als Objekt auszugeben (einen
gewrappten C-Pointer nennt man \emph{DATA}) gibt es die Funktion
\verb+Data_Wrap_Struct()+.

\begin{lstlisting}
Data_Wrap_Struct(klass, mark, free, ptr)
\end{lstlisting}

\noindent\verb+Data_Wrap_Struct()+ gibt das erstellte DATA-Objekt zurück. Das Argument
\verb+klass+ gibt die Klasse des DATA-Objekts an. \verb+mark+ ist die
Funktion, die sämtliche von der DATA referenzierten Ruby-Objekte zu
markieren, \verb+free+ ist die Funktion um den allokierten Speicher
freizugeben. Ist sie -1, wird das normale \verb+free()+ für den
Pointer benutzt.

Diese \verb+mark+/\verb+free+-Funktionen werden während der Ausführung
des Garbage Collectors aufgerufen. Keine Speicherplatzallokationen für
Objekte sind während dieser Phase erlaubt, also erstelle in ihr keine
Ruby-Objekte.

\trans{Ruby nutzt einen sogenannten »Mark-and-Sweep«-Garbage
  Collector, \dh zunächst wird eine Mark-Phase durchlaufen, in der
  alle in Gebrauch befindlichen Ruby-Objekte markiert werden. Das
  heißt für die \fverb{mark}-Funktion, dass sie sie in diesem Muster
  mitspielen muss und die Objekte, die das DATA-Objekt referenziert,
  markieren muss. Wenn sie dies nämlich nicht tut und das einzige
  Objekt ist, dass eine Referenz auf ein solches Objekt besitzt, wird
  der Garbage Collector dieses irrtümlich einsammeln. Dies ist die zweite
  Phase: Alle Objekte, die keine Markierung besitzen, werden vom
  Garbage Collector »entsorgt«, \dh ihr Speicher wird freigegeben. Da
  Ruby nicht weiß, wie es mit den von Ihnen allokierten Datenstrukturen
  Ihres DATA-Objekts umgehen soll, müssen Sie die
  \fverb{free}-Funktion bereitstellen, die der GC aufruft, wenn keine
  Referenz mehr auf Ihr Objekt exisitiert}

Sie können Allokation und Wrapping der Struktur in einem einzigen
Schritt durchführen:

\begin{lstlisting}
Data_Make_Struct(klass, type, mark, free, sval)
\end{lstlisting}

\noindent Dieses Makro gibt das allokierte Data-Objekt zurück, wobei
es den Pointer in die Struktur wrappt, die ebenfalls allokiert
wird. Insgesamt funktioniert das Makro etwa so:

\begin{lstlisting}
(sval = ALLOC(type), Data_Wrap_Struct(klass, mark, free, sval))
\end{lstlisting}

\noindent Die Argumente für klass, mark und free funktionieren wie
ihre Gegenstücke in \verb+Data_Wrap_Struct()+. Der Variablen
\verb+sval+ wird ein Pointer auf die allokierte Struktur zugewiesen,
daher sollte er vom angegebenen Typ sein.

Um den C-Pointer wieder aus dem DATA-Objekt auszupacken, verwenden Sie
\verb+Data_Get_Struct()+.

\begin{lstlisting}
Data_Get_Struct(obj, type, sval)
\end{lstlisting}

\noindent Der Variablen \verb+sval+ wird ein Pointer auf die Struktur
zugewiesen werden.

%%% Local Variables: 
%%% mode: latex
%%% TeX-master: "main"
%%% End: 
