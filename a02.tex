\chapter{Referenz zum Ruby-Extension-API}
\label{cha:api-referenz}

\section{Typen}
\label{sec:api:typen}

\begin{description}
\item[\texttt{VALUE}:] Das ist der Typ für das Ruby-Objekt. Die
  eigentlichen Strukturen wie \zB RString sind in der Datei
  \path{ruby.h} definiert. Um Werte aus diesen Strukturen abzufragen,
  sollten Sie Casting-Makros wie \verb+RSTRING(obj)+ benutzen.
\end{description}

\section{Variablen und Konstanten}
\label{sec:api:variablen-konstanten}

\begin{description}
\item[\texttt{Qnil}:] Konstant: \verb+nil+-Objekt.
\item[\texttt{Qtrue}:] Konstant: \verb+true+-Objekt
  (Standard-Wahrheitswert \trans{das bezieht sich auf Rubys Eigenart,
    alle Werte als wahr zu behandeln. \lstinline+true+ ist einfach
    \emph{nur} wahr, es enthält abgesehen von dieser Eigenschaft keine,
    aber auch gar keine Information. Dagegen ist \lstinline+27+ zwar
    auch wahr, enthält aber neben der Eigenschaft, wahr zu sein, auch
    noch die Information, dass es sich um den Wert 27 handelt, der
    durchaus noch einen mathematischen Sinn besitzt, der mit der
    Eigenschaft, wahr zu sein, nichts zu tun hat.}).
\item[\texttt{Qfalse}:] Konstant: \verb+false+-Objekt.
\end{description}

\section{Wrapping von C-Pointern}
\label{sec:api:c-pointer}

\begin{lstlisting}
Data_Wrap_Struct(VALUE klass, void (*mark)(),
                 void (*free)(), void *sval)
\end{lstlisting}

\noindent Wrappt einen C-Pointer in ein Ruby-Objekt hinein. Wenn
dieses Objekt weitere Ruby-Objekte referenziert, sollten jene während
des GC-Prozesses von der \verb+mark+-Funktion markiert werden.
Ansonsten sollte \verb+mark+ \verb+0+ sein. Wird das Objekt nirgendwo
mehr referenziert, so wird die \verb+free+-Funktion aufgerufen, um den
Pointer aufzulösen.

\begin{lstlisting}
Data_Make_Struct(klass, type, mark, free, sval)
\end{lstlisting}

\noindent Dieses Makro allokiert Speicher mithilfe von
\verb+malloc()+, weist diesen dann der Variablen \verb+sval+ zu und
gibt schließlich ein DATA-Objekt, welches den Pointer zum allokierten
Speicher enthält, zurück.

\begin{lstlisting}
Data_Get_Struct(data, type, sval)
\end{lstlisting}

\noindent Dieses Makro liest den Pointer aus einer DATA und weist ihn
der Variablen \verb+sval+ zu.

\section{Überprüfen von Datentypen}
\label{sec:api:datentypen}

\begin{lstlisting}
TYPE(value)
FIXNUM_P(value)
NIL_P(value)
void Check_Type(VALUE value, int type)
void Check_SafeStr(VALUE value)
\end{lstlisting}

\section{Konvertierung von Datentypen}
\label{sec:api-konvertieren}

\begin{lstlisting}
FIX2INT(value), INT2FIX(i)
FIX2LONG(value), LONG2FIX(l)
NUM2INT(value), INT2NUM(i)
NUM2UINT(value), UINT2NUM(ui)
NUM2LONG(value), LONG2NUM(l)
NUM2ULONG(value), ULONG2NUM(ul)
NUM2LL(value), LL2NUM(ll)
NUM2ULL(value), ULL2NUM(ull)
NUM2OFFT(value), OFFT2NUM(off)
NUM2SIZET(value), SIZET2NUM(size)
NUM2SSIZET(value), SSIZET2NUM(ssize)
NUM2DBL(value)
rb_float_new(f)
StringValue(value)
StringValuePtr(value)
StringValueCStr(value)
rb_str_new2(s)
\end{lstlisting}

\section{Klassen/Modul-Definition}
\label{sec:api-classmod}

\begin{lstlisting}
VALUE rb_define_class(const char *name, VALUE super)
\end{lstlisting}

\noindent Definiert eine neue Ruby-Klasse als Subklasse von
\verb+super+ \trans{Für Klassen, die einfach nur Subklasse von Object
  sein sollen, sollte man \fverb{rb\_cObject} als Superklasse nehmen}.

\begin{lstlisting}
VALUE rb_define_class_under(VALUE module,
                            const char *name,
                            VALUE super)
\end{lstlisting}

\noindent Definiert eine neue Ruby-Klasse als Subklasse von
\verb+super+ innerhalb des Moduls \verb+module+.

\begin{lstlisting}
VALUE rb_define_module(const char *name)
\end{lstlisting}

\noindent Definiert ein neues Ruby-Modul.

\begin{lstlisting}
VALUE rb_define_module_under(VALUE module, const char *name)
\end{lstlisting}

\noindent Definiert ein neues Ruby-Modul innerhalb eines bestehenden.

\begin{lstlisting}
void rb_include_module(VALUE klass, VALUE module)
\end{lstlisting}

\noindent Mischt \verb+module+ in \verb+klass+ ein, sofern dies nicht
bereits geschehen ist (in diesem Fall wird der Aufruf schlicht
ignoriert). \trans{Entspricht Module\#include.}

\begin{lstlisting}
void rb_extend_object(VALUE object, VALUE module)
\end{lstlisting}

\noindent Mischt \verb+module+ auf Instanzebene in \verb+object+
ein. \trans{Entspricht Object\#extend.}

\section{Definition globaler Variablen}
\label{sec:api-globals}

\begin{lstlisting}
void rb_define_variable(const char *name, VALUE *var)
\end{lstlisting}

\noindent Definiert eine globale Variable, auf die sowohl von Ruby als
auch von C aus zugegriffen werden kann. Wenn der Name ein Zeichen
enthält, das im Ruby-Symbol nicht erlaubt ist, können Ruby-Programme
nicht auf die Variable zugreifen \trans{in der Praxis lässt man, so
  man tatsächlich Ruby ausschließen will, meist das führende
  \$-Zeichen weg, was in Ruby 1.9 den gleichen Effekt hat}.

\begin{lstlisting}
void rb_define_readonly_variable(const char *name, VALUE *var)
\end{lstlisting}

\noindent Definiert eine nur-lesbare globale Variable. Funktioniert
ansonsten genau wie \verb+rb_define_variable()+.

\begin{lstlisting}
 void rb_define_virtual_variable(const char *name,
                                 VALUE (*getter)(),
                                 VALUE (*setter)())
\end{lstlisting}

\noindent Definiert eine virtuelle Variable, deren Verhalten nur durch
ein Paar von C-Funktionen bestimmt wird. Die \verb+getter+-Funktion
wird aufgerufen, wenn die Variable referenziert wird, \verb+setter+
wird bei Zuweisungen aufgerufen. Die Prototypen für
\verb+getter+/\verb+setter+ sind:

\begin{lstlisting}
VALUE getter(ID id)
oid setter(VALUE val, ID id)
\end{lstlisting}

\noindent Die \verb+getter+-Funktion muss den Wert für den Aufruf
zurückgeben.

\begin{lstlisting}
void rb_define_hooked_variable(const char *name,
                              VALUE *var,
                              VALUE (*getter)(),
                              VALUE (*setter)())
\end{lstlisting}

\noindent Definiert eine \emph{Hook}-Variable, also eine virtuelle
Variable, die gleichzeitig eine C-Variable ist. Der \verb+getter+ wird
als

\begin{lstlisting}
VALUE getter(ID id, VALUE *var)
\end{lstlisting}

\noindent aufgerufen und gibt einen neuen Wert zurück. Der
\verb+setter+ als:

\begin{lstlisting}
void setter(VALUE val, ID id, VALUE *var)
\end{lstlisting}

Damit der GC globale C-Variablen, die Ruby-Objekte referenzieren,
nicht irrtümlich einesammelt, müssen diese markiert werden:

\begin{lstlisting}
void rb_global_variable(VALUE *var)
\end{lstlisting}

\noindent weist den GC an, diese zu schützen.

\section{Konstantendefinition}
\label{sec:api-const}

\begin{lstlisting}
void rb_define_const(VALUE klass, const char *name, VALUE val)
\end{lstlisting}

\noindent Definiert eine Konstante innerhalb einer Klasse/eines
Moduls.

\begin{lstlisting}
void rb_define_global_const(const char *name, VALUE val)
\end{lstlisting}

\noindent Definiert eine globale Konstante. Das ist einfach das
gleiche wie:

\begin{lstlisting}
rb_define_const(rb_mKernel, name, val)
\end{lstlisting}

\section{Methodendefinition}
\label{sec:mapi-meth}

\begin{lstlisting}
rb_define_method(VALUE klass,
                 const char *name,
                 VALUE (*func)(),
                 int argc)
\end{lstlisting}

\noindent Definiert eine Methode für die angegebene
Klasse. \verb+func+ ist die aufzurufende Funktion, \verb+argc+ die
Anzahl der Argumente. Wenn \verb+argc+ -1 ist, erhält die Funktion
drei Argumente: \verb+argc+ (Argumentanzahl), \verb+argv+ (Argumente
als C-\verb+VALUE+-Array) und \verb+self+ (der Sender). Ist
\verb+argc+ -2, erhält die Funktion zwei Argumente, \verb+self+ und
\verb+args+, wobei \verb+args+ ein Ruby-Array der Methoden-Argumente
ist.

\begin{lstlisting}
rb_define_private_method(VALUE klass,
                         const char *name,
                         VALUE (*func)(),
                         int argc)
\end{lstlisting}

\noindent Definiert eine private Methode für eine Klasse. Die
Argumente sind die gleichen wie für \verb+rb_define_method()+.

\begin{lstlisting}
rb_define_singleton_method(VALUE klass,
                           const char *name,
                           VALUE (*func)(),
                           int argc)
\end{lstlisting}

\noindent Definiert eine Singleton-Methode
\trans{bzw. Klassenmethode}. Die Argumente sind die gleichen wir für
\verb+rb_define_method()+.

\begin{lstlisting}
rb_scan_args(int argc,
             VALUE *argv,
             const char *fmt,
             ...)
\end{lstlisting}

\noindent Bindet die Argumente aus \verb+argc+ und \verb+argv+ and die
durch den Format-String angegebenen \VALUE-Referenzen. Das Format kann
in \acronym{ABNF} wie folgt beschrieben werden:

\begin{verbatim}
num-of-leading-mandatory-args  := DIGIT ; Die Anzahl der führenden
                                        ; erforderlichen Argumente.
num-of-optional-args           := DIGIT ; Die Anzahl der optionalen
                                        ; Argumente.
sym-for-variable-length-args   := "*"   ; Zeigt an, dass variable
                                        ; Argumente als Ruby-Array
                                        ; dargestellt werden.
num-of-trailing-mandatory-args := DIGIT ; Die Anzahl der abschließenden
                                        ; erforderlichen Argumente.
sym-for-option-hash-arg        := ":"   ; Zeigt an, dass ein Optionshash
                                        ; genutzt wird, falls das letzte
                                        ; Argument ein Hash ist oder mittels
                                        ; #to_hash in einen solchen konver-
                                        ; tiert werden kann. Ist das letzte
                                        ; Argument nil, wird es nur dann
                                        ; übernommen, falls keine Doppeldeutig-
                                        ; keit mit der Übernahme als leeres
                                        ; Optionshash auftreten können, d.h.
                                        ; '*' nicht angegeben wurde und die
                                        ; übergebene Argumentanzahl größer
                                        ; als erforderlich ist.
sym-for-block-arg              := "&"   ; Zeigt an, dass ein Iterator-
                                        ; block, falls übergeben, ge-
                                        ; fangen werden soll.
\end{verbatim}

\noindent Beispielsweise bedeuetet \verb+"12"+, dass die Methode
mindestens ein Argument erfordert, und höchstens drei (1+2). Daher
müssen dem Formatstring Referenzen (Pointer) zu drei Variablen
\trans{vom Typ \VALUE} folgen, welche auf die erkannten Argumente
gesetzt werden. Im Falle ausgelassener Argumente, werden die
entsprechenden Variablen auf \verb+Qnil+ gesetzt. Auch kann
\verb+NULL+ anstelle eine Variablenreferenz übergeben werden, was zur
Ignoranz des jeweiligen erkannten Arguments führt.

Rückgabewert ist die Anzahl übergebener Argumente,
abzüglich eines Optionshashes und des Iteratorblocks.


%%% Local Variables: 
%%% mode: latex
%%% TeX-master: "main"
%%% End: 
