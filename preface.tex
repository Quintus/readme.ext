\section*{Vorwort zur zweiten Auflage}
\addcontentsline{toc}{section}{Vorwort zur zweiten Auflage}
\label{sec:vorwort2}

Knappe zwei Jahre nach dem Erscheinen der ersten Auflage dieser
Übersetzung habe ich mich nun dazu entschlossen, die damalige Version
den aktuellen Gegebenheiten anzupassen. Das schließt zum einen die
Angleichung des Inhalts an die vom Core-Team durchgeführten Änderungen
an der Quelldatei \path{README.EXT} in den Ruby-Quellen, zum anderen
aber auch die inhaltliche wie orthografisch-grammatikalische
Überarbeitung des bisherigen Textes mit ein. Auch das gesamte
Erscheinungsbild der zweiten Auflage gegenüber der ersten sollte
wesentlich harmonischer wirken, bin ich doch in dieser Zeit um die
Kenntnis des Schriftsatzsystems \LaTeX{} und des damit einhergehenden
typografischen Grundwissens reicher geworden.

Grund für die neue Auflage ist zum einen die schon eingangs erwähnte
Anpassung des Inhalts an die neuen Gegebenheiten, ist doch die in der
vorangegangenen Auflage beschriebene Ruby-Version 1.9.1 mehr oder
minder als Entwicklerversion anzusehen\todo{Referenz!} und erst 1.9.2
stabilisierte den neugeschaffenen Ruby-Interpreter, der nun in der
ausgereiften Verison 1.9.3 vorliegt (ich möchte an dieser Stelle alle
Leute, die noch Ruby 1.8 benutzen, zu einer Aktualisierung auf 1.9
ermutigen!). Darüber hinaus wollte ich gerne einige zusätzliche
Kenntnisse, die ich im Laufe der Zeit bei der Implementation der ein
oder anderen C-Extension oder der Betrachtung von anderen Entwicklern
erstellter C-Extensions gesammelt habe mit einfließen
lassen. Schließlich ist ein dritter Grund aufzuführen: Mit dieser
Auflage findet sich der Quellcode des Dokuments erstmalig auf
\name{GitHub}, genaugenommen unter \todo{Link einpflegen!}, wodurch
eine einfache Mitarbeit aller Interessierten nun möglich und auch
ausdrücklich erwünscht wird. Wenn der geneigte Leser also die ein oder
andere Ergänzung für erforderlich hält oder eine stilistisch schlecht
gelungene Passage nachbessern möchte, ist er herzlich dazu eingeladen,
seine Verbesserungsvorschläge und Kritiken als Ticket unter \todo{Link
  einpflegen!} zu melden. Selbstverständlich steht es Ihnen offen, das
Projekt zu forken und die Verbesserungen selbst vorzunehmen. Ich werde
alle das Dokument fördernde Pull-Requests dankend entgegen nehmen.

Abgesehen von inhaltlichen und äußerlichen Korrekturen sind in diese
Auflage wie schon genannt die neuen Änderungen an der Quelldatei
\path{README.EXT} eingeflossen. Die Art und Weise der Präsentation hat
sich zur (hoffentlich) größeren Lesbarkeit gewandelt und der größeren
Akzeptanz wegen habe ich mich entschieden, Sie, den Leser, nun in
dieser Form statt mit dem in der Vorgängerversion verwendeten
informalen »Du« anzusprechen.

So, jetzt habe ich aber genug gefaselt. In der Hoffnung, dass dieses
Dokument dem geneigten Leser möglichst nützlich erscheint, gebe ich
nun ohne weitere Umschweife und größeres Herumgerede oder anderweitige
Verzögerungen den Weg für ein hoffentlich eher entspanntes als
gespanntes Lesen des Textes frei.

\begin{flushright}
  Marvin Gülker (alias \emph{Quintus}),\\
  am \today
\end{flushright}

%%% Local Variables: 
%%% mode: latex
%%% TeX-master: "main"
%%% End: 
