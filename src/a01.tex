\chapter{Übersicht über die Ruby-Quelldateien}
\label{cha:ruby-quellen-uebersicht}

Hier finden Sie eine Übersicht über die Organisation von Rubys eigenem
Quellcode.

%\begin{longtable}{p{0.33\textwidth}p{0.66\textwidth}}
\begin{longtable}{>{\ttfamily}p{0.33\textwidth}p{0.66\textwidth}}
  \caption{Übersicht über die Ruby-Quelldateien}\label{tab:rubyquellen}\\
  \textbf{\rmfamily Datei} & \textbf{Beschreibung}\\ \hline
\endfirsthead
  \textbf{\rmfamily Datei} & \textbf{Beschreibung}\\ \hline
\endhead
  \hline \multicolumn{2}{r}{\emph{Fortsetzung auf nächster Seite}}
\endfoot
  \hline
\endlastfoot
  \multicolumn{2}{c}{\textbf{Sprachkern von Ruby}}\\
  \hline

  class.c    & Klassen und Module\\
  error.c    & Exception-Klassen und der Exception-Mechanismus.\\
  gc.c       & Speicherverwaltung\\
  load.c     & Laden von Programmbibliotheken\\
  object.c   & Objekte\\
  variable.c & Variablen und Konstanten\\

  \hline
  \multicolumn{2}{c}{\textbf{Ruby-Syntax-Parser}}\\
  \hline

  parse.y               & YACC-Grammatik \transmark\\
  $\rightarrow$ parse.c & Automatisch generiert\\
  keywords              & Reservierte Schlüsselwörter\\
  $\rightarrow$ lex.c   & Automatisch generiert\\

  \hline
  \multicolumn{2}{c}{\textbf{Ruby-Evaluierer (alias YARV)}}\\
  \hline

  compile.c               & \\
  eval.c                  & \\
  eval\_error.c           & \\
  eval\_jump.c            & \\
  eval\_safe.c            & \\
  insns.def               & Definition der VM-Anweisungen\\
  iseq.c                  & Implementation von \verb+VM::ISeq+\\
  thread.c                & Thread-Verwaltung und Kontextänderungen\\
  thread\_win32.c         & Thread-Implementation \trans{für Windows}\\
  thread\_pthread.c       & Dito \trans{für unixoide Systeme}\\
  vm.c                    & \\
  vm\_dump.c              & \\
  vm\_eval.c              & \\
  vm\_exec.c              & \\
  vm\_insnhelper.c        & \\
  vm\_method.c            & \\
                          & \\
  opt\_insns\_unif.def    & Vereinheitlichung der Anweisungen \\
  opt\_operand.def        & Definitionen zur Optimierung \\
  $\rightarrow$ insn*.inc & Automatisch generiert\\
  $\rightarrow$ opt*.inc  & Automatisch generiert\\
  $\rightarrow$ vm.inc    & Automatisch generiert\\

  \hline
  \multicolumn{2}{c}{\textbf{Engine für Reguläre Ausdrücke (Oniguruma)}}\\
  \hline

  regex.c     & \\
  regcomp.c   & \\
  regenc.c    & \\
  regerror.c  & \\
  regexec.c   & \\
  regparse.c  & \\
  regsyntax.c & \\

  \hline
  \multicolumn{2}{c}{\textbf{Nützliche Funktionen}}\\
  \hline

  debug.c    & Debugging-Symbole für den C-Debugger\\
  dln.c      & Dynamisches Laden\\
  st.c       & Allgemeine Hashtabelle\\
  strftime.c & Zeiten formatieren\\
  util.c     & Sonstige Dinge\\

  \hline
  \multicolumn{2}{c}{\textbf{Implementation des Ruby-Interpreters}}\\
  \hline

  dmyext.c        & \\
  dmydln.c        & \\
  dmyencoding.c   & \\
  id.c            & \\
  inits.c         & \\
  main.c          & \\
  ruby.c          & \\
  version.c       & \\

  gem\_prelude.rb & \\
  prelude.rb      & \\

  \hline
  \multicolumn{2}{c}{\textbf{Klassenbibliothek}}\\
  \hline

  array.c                         & \verb+Array+\\
  bignum.c                        & \verb+Bignum+\\
  compar.c                        & \verb+Comparable+\\
  complex.c                       & \verb+Complex+\\
  cont.c                          & \verb+Fiber+, \verb+Continuation+\\
  dir.c                           & \verb+Dir+\\
  enum.c                          & \verb+Enumerable+\\
  enumerator.c                    & \verb+Enumerator+\\
  file.c                          & \verb+File+\\
  hash.c                          & \verb+Hash+\\
  io.c                            & \verb+IO+\\
  marshal.c                       & \verb+Marshal+\\
  math.c                          & \verb+Math+\\
  numeric.c                       & \verb+Numeric+, \verb+Integer+, \verb+ixnum+, \verb+Float+\\
  pack.c                          & \verb+Array#pack+, \verb+String#unpack+\\
  proc.c                          & \verb+Binding+, \verb+Proc+\\
  process.c                       & \verb+Process+\\
  random.c                        & Zufallszahlen\\
  range.c                         & \verb+Range+\\
  rational.c                      & \verb+Rational+\\
  re.c                            & \verb+Regexp+, \verb+MatchData+\\
  signal.c                        & \verb+signal+\\
  sprintf.c                       & \verb+Kernel#sprintf+ \transmark\\
  string.c                        & \verb+String+\\
  struct.c                        & \verb+Struct+\\
  time.c                          & \verb+Time+\\
                                  & \\
  defs/known\_errors.def          & \verb+Errno::*+-Exception-Klassen\\
  $\rightarrow$ known\_errors.inc & Automatisch generiert\\

  \hline
  \multicolumn{2}{c}{\textbf{Multilingualisation}}\\
  \hline

  encoding.c  & \verb+Encoding+\\
  transcode.c & \verb+Encoding::Converter+\\
  enc/*.c     & Encoding-Klassen\\
  enc/trans/* & Tabellen zum Mapping von Codepoints \trans{(Unicode-Codepoints)}.\\

  \hline
  \multicolumn{2}{c}{\textbf{Implementation des Goruby-Interpreters}}\\
  \hline

  goruby.c                      & \\
  golf\_prelude.rb              & Goruby-spezifische Programmbibliotheken\\
  $\rightarrow$ golf\_prelude.c & Automatisch generiert\\
\end{longtable}

%%% Local Variables: 
%%% mode: latex
%%% TeX-master: "main"
%%% End: 
